\section{Trust}

In contrast to Dissent, it is difficult to quantify the trust requirements for Heresy. In Dissent, the system guarantees anonymity as long as one server is honest. In Heresy, if one server per Dissent group is honest, Heresy achieves the same anonymity. However, Heresy still achieves a certain degree of anonymity even if some Dissent groups are completely dishonest.

When each Dissent group has one honest server, the size of the anonymity group is $n$ where $n$ is the number of leaves, or users, in the Heresy tree. For each dishonest Dissent group, that is, every server is dishonest, the size of the anonymity group decreases. The extent to which the anonymity decreases depends on the groups location in the tree. 

If an entire Dissent group is dishonest, then an attacker knows which subtree (or leaf) the output message corresponds to. As the messages are onion-encrypted for each layer, an attacker can only correlate the message between groups which are entirely dishonest. This implies that as long as one group along the path from the root to a particular end-client has at least one honest server, not all is lost. In particular, the anonymity group is at least $k^{l-q}$where $q$ is level of the honest group, with the root-level being labelled 0. In short, the system can achieve very large anonymity despite many dishonest servers. 
